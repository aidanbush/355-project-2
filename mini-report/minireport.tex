\documentclass[12pt]{article}
\usepackage{fullpage}
\usepackage[top=2cm, bottom=4.5cm, left=2.5cm, right=2.5cm]{geometry}
\usepackage{fancyhdr}

\pagestyle{fancyplain}
\headheight 35pt
\lhead{Aidan Bush, David Dowie, Ben Ha}
\rhead{CMPT 355 \\ \today}
\headsep 1.5em

\begin{document}

\section*{Project 2-Konane}

\subsection*{Minmax Implementation}
In our approach we will use Alpha-Beta pruning with iterative deepening.
Iterative deepening is used so that we are able better work within the time limit and explore more of a breadth of possible moves.
We will also be sorting potential moves in order of highest to lowest for our MAX function and visa versa for our MIN function.

\subsection*{Evaluation Function}

\subsection*{Representation of State Space}
To represent our state space we will use an array of 8 16bit integers.
Every two bits is a square on the board, and every entry in in the array will be a line.
We do this to reduce our overall memory usage.
This memory model should increase the speed of the program as it will have to access less memory.
% TODO add more and figure

\end{document}
