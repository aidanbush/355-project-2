\documentclass[12pt]{article}
\usepackage{fullpage}
\usepackage[top=2cm, bottom=4.5cm, left=2.5cm, right=2.5cm]{geometry}
\usepackage{fancyhdr}

\pagestyle{fancyplain}
\headheight 35pt
\lhead{Aidan Bush, David Dowie, Ben Ha}
\rhead{CMPT 355 \\ \today}
\headsep 1.5em

\begin{document}

\section*{Project 2-Konane}

\subsection*{Minmax Implementation}
In our approach we will use Alpha-Beta pruning with iterative deepening.
Iterative deepening is used so that we are able better work within the time limit and explore more of a breadth of possible moves.
We will also be sorting potential moves in order of highest to lowest for our MAX function and visa versa for our MIN function.

\subsection*{Evaluation Function}
For our evaluation function we have considered various ways of evaluating the given state in a way that would help us generate a optimal solution. Some evaluations that we have considered are as follows:
\begin{itemize}
\item \textbf{Number of moves Heuristic Evaluation} - 
Takes in the given state and calculates the number of moves available to max for that given state. 
With this evaluation it will always provide a higher value for the state that has more available moves. 
This evaluation would not be a good choice for implementing into our minmax algorithm as min choosing the smaller heuristic value would not necessarily be the optimal for that node. 
This would lead to a suboptimal choice as the heuristic only looks at the number of moves available for max, which leads to min choosing the value that minimizes the number of choices for max. 
Although min would be minimizing the available moves for max, this does not necessarily mean that we also are maximizing the available moves for min.

\end{itemize}
\subsection*{Representation of State Space}
To represent our state space we will use an array of 8 16bit integers.
Every two bits is a square on the board, and every entry in in the array will be a line.
We do this to reduce our overall memory usage.
This memory model should increase the speed of the program as it will have to access less memory.
% TODO add more and figure

\subsection*{Programming language}
We will be using C for imporved efficiency
\end{document}
